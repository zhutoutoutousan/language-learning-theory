\section{Theoretical Framework}
\subsection{Krashen's Five Hypotheses}
\subsubsection{The Acquisition-Learning Hypothesis}
Krashen's distinction between acquisition (subconscious process) and learning (conscious process) \citep{krashen2013second} remains fundamental to understanding how adults approach language learning. In digital environments, this distinction informs the design of activities that promote natural language acquisition while providing structured learning opportunities.

\subsubsection{The Monitor Hypothesis}
The Monitor Hypothesis explains how learned knowledge acts as an editor for acquired language \citep{krashen1982principles}. Modern applications can leverage this understanding by providing appropriate feedback mechanisms that support, rather than hinder, natural communication.

\subsubsection{The Natural Order Hypothesis}
Research has shown that grammatical structures are acquired in a predictable order \citep{dulay1974natural}. This understanding is crucial for designing progressive learning paths that align with natural acquisition sequences, particularly in gamified environments where skill progression must feel natural and achievable.

\subsubsection{The Input Hypothesis}
The concept of comprehensible input (i+1) guides the development of content that is slightly above a learner's current level \citep{krashen2013second}. Digital platforms can dynamically adjust content difficulty based on learner performance, ensuring optimal challenge levels.

\subsubsection{The Affective Filter Hypothesis}
The role of emotional factors in language acquisition \citep{krashen1982principles} is particularly relevant in digital learning environments. Gamification elements can help lower the affective filter by reducing anxiety and increasing motivation through positive reinforcement and achievable challenges.

\subsection{Modern Extensions to Traditional Theory}
Recent research has expanded upon Krashen's theories, incorporating insights from cognitive science and digital learning \citep{ellis2015understanding}. These extensions consider how modern technology can enhance traditional acquisition methods while addressing contemporary learning challenges.

\subsection{Neurolinguistic Perspectives}
Current neurolinguistic research provides valuable insights into how the brain processes language learning at different ages \citep{brown2014principles}. This understanding informs the development of age-appropriate learning strategies and helps address specific challenges faced by adult learners. 